\documentclass[paper=a4, fontsize=17pt]{article}

\usepackage[russian]{babel}
\usepackage{scrextend}
\usepackage[utf8x]{inputenc}
\usepackage[T1,T2A]{fontenc}
\usepackage[left=1.5cm,right=1.5cm,top=1.5cm,bottom=1.5cm,bindingoffset=0cm]{geometry}
\usepackage[pdftex]{graphicx}
\usepackage{amsmath}
\usepackage{mathtools}
\usepackage{ulem}
\usepackage{mathrsfs}
\usepackage{amsfonts}
\usepackage{dsfont}
\usepackage{amssymb}
\usepackage{cmap}
\usepackage{hyperref}
\usepackage{graphics}
\usepackage{amssymb}
\usepackage{calrsfs}
\DeclareMathAlphabet{\pazocal}{OMS}{zplm}{m}{n}
\DeclareMathOperator*{\esssup}{ess\, sup}


\parindent=0cm

\title{Определения по матану, семестр 4}

\begin{document}
	\maketitle
	\tableofcontents
	\newpage
	
\section{Теорема о вложении пространств $L^p$}

$ \mu E < +\infty, ~ 1 \leq s < r \leq + \infty$

\textbf{\emph{Тогда:}} 

\begin{enumerate}
	\item $ L_r(E, \mu)  \subset \L_s(E, \mu)$
	\item $ \forall f - $ измеримы $ ~ ||f||_s \leq \mu E^{1/s - 1/r} ||f||_r$
\end{enumerate}	

\section{Теорема о сходимости в $L_p$ и по мере}
$ 1 \leq p < +\infty, ~$
$ f_n \in L_p(\mathbb{X}, \mu)$

\textbf{\emph{Тогда:}}

\begin{enumerate}
	
	
	\item \begin{itemize}
		\item $ f \in L_p $
		\item $ f_n \rightarrow f $ в $ L_p $
	\end{itemize} 
	\textbf{Тогда:} $ f_n \stackrel{\mu}{\Rightarrow} f $ (по мере)
		
	\item \begin{itemize}
		\item $ f_n \stackrel{\mu}{\Rightarrow} f $ (либо если $ f_n \rightarrow f $  почти везде)
		\item $ |f_n| \leq g $ почти при всех $ n ~ , ~ g \in L_p $
	\end{itemize} 
	\textbf{Тогда:} $ f_n \rightarrow f $ в $ L_p $
\end{enumerate}

\section{Полнота $L_p$}
$ L_p(E, \mu) ~ ~ 1 \leq p < \infty $ -- полное 

То есть любая фундаментальная последовательность сходиться по норме $ ||f||_p $.

$$ \forall \varepsilon > 0 ~ \exists N ~ \forall n, k ~ ||f_n - f_k|| < \varepsilon \Rightarrow \exists f \mid ~ ||f_n - f|| \rightarrow 0 $$


\section{Лемма Урысона}

$F_0, F_1 - $ два непересекающихся замкнутых множества из $\mathbb{R}^m$

Тогда $\exists f: \mathbb{R}^m \to \mathbb{R}$ (непрырывная)$: f|_{F_0}=0, f|_{F_1}=1$

\section{Плотность в $L_p$ непрерывных финитных функций}

$\forall p: 1 \leqslant p < +\infty \quad C_0$ всюду плотно в $L^p(R^m)$

\section{Теорема о непрерывности сдвига}

$f_n(x) = f(x + h)$

* $f$ - равномерно непрерывна на $\mathbb{R}^m \Rightarrow \displaystyle\lim_{h \to 0} \| f_n - f \|_\infty = 0$

* $1 \leqslant p < +\infty \quad f \in L^p (\mathbb{R}^m) \Rightarrow \displaystyle\lim_{h \to 0} \| f_n - f \|_p = 0$

* $f \in \widetilde{C}[0, T] \Rightarrow \displaystyle\lim_{h \to 0} \| f_n - f \|_\infty = 0$

* $1 \leqslant p < +\infty \quad f \in L^p[0, T] \Rightarrow \lim\limits_{h \to 0} \| f_n - f \|_p = 0$

\section{Теорема об интеграле с функцией распределения}
$(\mathbb{R}, B, X)$ \\
$f:\mathbb{R}\rightarrow\mathbb{R}, f \ge 0,$ изм. по Борелю, п.в. конечн.\\
$h: X \rightarrow \overline{\mathbb{R}}$ с функцией распределения $H(t)$\\ 


$\mu_H$ -- мера Бореля-Стилтьеса (мера Лебега-Стилтьеса на $B$)\\


\emph{Тогда:} $\int\limits_X f(h(x))~d\mu(x) = \int\limits_{\mathbb{R}}f(t)~d\mu_{H}(t)$

\section{Теорема о свойствах сходимости в гильбертовом пространстве}
\begin{enumerate}
	\item $x_n \rightarrow x, y_n \rightarrow y \Rightarrow <x_n, y_n> \rightarrow <x, y>$
	
	\item $\sum x_k$ сходится, тогда $\forall y: \sum <x_k, y> = <\sum x_k, y>$
	
	\item $\sum x_k$ - ортогональный ряд, тогда $\sum x_k$ - сх $\Leftrightarrow \sum |x_k|^2$ сходится, при этом $|\sum x_k|^2 = \sum |x_k|^2$
	
\end{enumerate}

\section{Теорема о коэффициентах разложения по ортогональной системе}

$\{e_k\}$ {{---}} ортогональная система в $\mathds{H},\ x \in \mathds{H}, x = \sum\limits_{k=1}^{+\infty} c_k \cdot e_k$

\emph{Тогда:}
\begin{enumerate}

	\item $\{e_k\}$ {{---}} Л.Н.З.
	
	\item $c_k = \dfrac{<x, e_k>}{||e_k||^2}$
	
	\item $c_k \cdot e_k$ {{---}} проекция $x$ на прямую $\{te_k, t \in \mathbb{R}(\mathbb{C})\}$\\ 
	Иными словами $x = c_k \cdot e_k + z$, где $z \bot e_k$

\end{enumerate}

\section{Теорема о свойствах частичных сумм ряда Фурье. Неравенство Бесселя}

$\{e_k\}$ {{---}} ортогональная система в $\mathds{H},\ x \in \mathds{H}, n \in \mathbb{N}$

$S_n = \sum\limits_{k=1}^{n} c_k(x)e_k,\ \pazocal{L} = Lin(e_1, e_2, ...e_n) \subset \mathds{H}$

\emph{Тогда:}

\begin{enumerate}

	\item $S_n$ {{---}} орт. проекция $x$ на пр-во $\pazocal{L}$. Иными словами $x = S_n + z,\ z \bot \pazocal{L}$
	
	\item $S_n$ {{---}} наилучшее приближение $x$ в $\pazocal{L}\ (||x - S_n|| = \min\limits_{y \in \pazocal{L}} ||x - y||)$	

	\item $||S_n|| \leqslant ||x||$

\end{enumerate}

\section{Теорема Рисса -- Фишера о сумме ряда Фурье. Равенство Парсеваля}
$\{e_k\}$ -- орт. сист. в $\mathds{H}$, $x \in \mathds{H}$\\

\emph{Тогда}:
\begin{enumerate}
	\item Ряд Фурье $\sum\limits_{k=1}^{+\infty} c_k(x) e_k$ сх-ся в $\mathds{H}$
	\item $x =\sum\limits_{k=1}^{+\infty} c_k e_k + z \Rightarrow \forall k \ z \bot e_k$
	\item $x =\sum\limits_{k=1}^{+\infty} c_k e_k \Leftrightarrow \sum\limits_{k=1}^{+\infty} \vert c_k \vert^2 \|e_k\|^2=\|x\|^2$
\end{enumerate}

\section{Теорема о характеристике базиса}

$\{e_k\}$ -- орт. сист. в $\mathds{H}$\\

\emph{Тогда} эквивалентны следующие утверждения:
\begin{enumerate}
	\item $\{e_k\}$ -- базис
	\item $\forall x, y \in \mathds{H} \quad \langle x, y \rangle = \sum c_k(x)\overline{c_k(y)}\|e_k\|^2$ (обобщенное уравнение замкнутости)
	\item $\{e_k\}$ - замкн.
	\item $\{e_k\}$ - полн.
	\item $Lin(e_1, e_2, \mathellipsis)$ - плотна в $\mathds{H}$ 
\end{enumerate}

\section{Лемма о вычислении коэффициентов тригонометрического ряда}

Пусть $S_n \rightarrow f$ в $L_1[-\pi, \pi]$\\ 

\emph{Тогда}:

$a_k = \frac{1}{\pi}\int\limits_{-\pi}^{\pi}f(x)coskx\ dx \quad k = 0, 1, 2, \mathellipsis$

$b_k = \frac{1}{\pi}\int\limits_{-\pi}^{\pi}f(x)sinkx\ dx \quad k = 0, 1, 2, \mathellipsis$

$c_k = \frac{1}{2\pi}\int\limits_{-\pi}^{\pi}f(x)e^{-ikx}\ dx \quad k = 0, 1, 2, \mathellipsis$

\section{Теорема Римана--Лебега}

$E \subset \mathbb{R}$ — измеримо, $f \in L^1(E)$

Тогда $\displaystyle\int\limits_E {f(x) \cdot e^{ikx} \; dx} \xrightarrow[k \to \infty]{} 0$ (То же самое можно и с $\cos {x}$ и $\sin {x}$ вместо $e^{ikx}$)

\section{Принцип локализации Римана}

$f, g \in L^1[-\pi, \pi] \quad x_0 \in [-\pi, \pi] \quad \exists \delta > 0$

$f(x) = g(x) $ при $ x \in (x_0 - \delta, x_0 + \delta)$

Тогда $S_n(f, x_0) - S_n(g, x_0) \xrightarrow[n \to +\infty]{} 0$

\section{Признак Дини. Следствия}

$f \in L^1[-\pi, \pi] \quad x_0 \in [-\pi, \pi] $
$S \in \mathbb R $ (или $\mathbb C$)

$\int\limits_0^{\pi} \frac{\lvert f(x_0 + t) - 2S + f(x_0 - t) \rvert}{t} dt < +\infty$

\emph{Тогда:} $S_n(f, x_0) \rightarrow S$

\emph{Следствие:}
$f \in L^1[-\pi, \pi] \quad x_0 \in [-\pi, \pi] $\\
Существуют 4 конечный предела $\alpha_{\pm} = \lim\limits_{t\rightarrow \pm 0} \frac{f(x_0 + t) - f(x_0 \pm 0)}{t}$

\emph{Тогда:} ряд Фурье сходится в $x_0$ к $S = \frac{1}{2}(f(x_0 + 0) + f(x_0 - 0))$

\emph{Следствие:}
$f \in L^1[-\pi, \pi] $ - непр. в $x_0$\\
Существует конечн. $f_{+}'(x_0), f_{-}'(x_0)$ и $\lim\limits_{t\rightarrow \pm 0} \frac{f(x_0 + t) - f(x_0 \pm 0)}{t}$

\emph{Тогда:} $S_n(f,x_0) \rightarrow f(x_0)$
 

\section{Корректность определения свертки}
Свертка -- корректно заданная функция из $ L_1([-\pi, \pi]) $

\section{Свойства свертки функции из $L^p$ с функцией из $L^q$}


$f \in L^p \quad k \in L^q[-\pi, \pi] \quad \left(\dfrac{1}{p} + \dfrac{1}{q} = 1 \right) \quad 1 \leqslant p < +\infty$

Тогда $f * k$ - непрерывна на $[-\pi, \pi]$

$\|f * k \|_1 \leqslant \|f\|_p * \|k\|_q$

\section{Формула Грина}

$\mathbb R^2$ — ориент. с помощью нумерации координат.

$D \subset \mathbb R^2$ — компактное, связное, односвязное, с $C^2$-гладкой границей.

$(P, Q)$ — гладкое векторное поле.

Пусть граница $D (\partial D)$ ориентирована согласованно с ориентацией плоскости.

Тогда $\displaystyle\int_{\partial D} P \,dx + Q \,dy = \displaystyle\iint_D \left(\dfrac{\partial Q}{\partial x} - \dfrac{\partial P}{\partial y}\right) dx\, dy$

\section{Формула Стокса}

$D \subset \mathbb R^3$ — простая гладкая поверхность в $\mathbb R^3$,
$\partial D$ — $C^2$-гладкая кривая,

$n_0$ — сторона поверхности; ориентированы согласованно с $\partial D$<br>
$(P,Q,R)$ — гладкое векторное поле на $D$. Тогда:

$\displaystyle\int_{\partial D} P dx + Q dy + R dz =$

$= \displaystyle\iint_D (R'_y - Q'_z) \;dy dz + (P'_z - R'_x) \;dz dx + (Q'_x - P'_y) \;dx dy$

\section{Формула Гаусса--Остроградского}

$D \subset \mathbb R^3 \quad \partial D$ — ориент. полем внешних нормалей

$(P, Q, R)$ — гл. век. поле в $D$. Тогда


 $\displaystyle\iint\limits_{\partial D} P \,dy\,dz + Q \,dz\,dx + R \,dx\,dy = \iiint\limits_D (P'_x + Q'_y + R'_z)\,dx\,dy\,dz$

\section{Соленоидальность бездивергентного векторного поля}
$A \in C^1$ \\
$A$ - соленоидально $\Leftrightarrow$ $div A = 0$

\end{document}
