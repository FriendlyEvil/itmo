\documentclass[paper=a4, fontsize=14pt]{report}

\usepackage[russian]{babel}
\usepackage{scrextend}
\usepackage[utf8x]{inputenc}
\usepackage[T1,T2A]{fontenc}
\usepackage[left=1.5cm,right=1.5cm,top=1.5cm,bottom=1.5cm,bindingoffset=0cm]{geometry}
\usepackage[pdftex]{graphicx}
\usepackage{amsmath}
\usepackage{mathtools}
\usepackage{ulem}
\usepackage{mathrsfs}
\usepackage{amsfonts}
\usepackage{dsfont}
\usepackage{amssymb}
\usepackage{cmap}
\usepackage{hyperref}
\usepackage{tikz}

\DeclareMathOperator*{\esssup}{ess\, sup}

\parindent=0cm
\linespread{1.2}
\author{}

\title{Первый коллоквиум, семестр 4}

\begin{document}
\maketitle
\tableofcontents
\newpage

\chapter{Определения}
	\section{Произведение мер}
	$<X, \mathds{A}, \mu>$, $<Y, \mathds{B}, \nu>$ - пространства с мерой.\\
	$\mu$, $\nu$ - $\sigma$-конечные меры.\\
	$\mathds{A} \times \mathds{B} = \{A\times B \subset X \times Y : A \in \mathds{A}, B \in \mathds{B} \}$ \\
	$m_0 : \mathds{A} \times \mathds{B} \rightarrow \overline{\mathds{R}}$\\ $m_0(A \times B) = \mu A \cdot \nu B$

	$m$ - называется произведением мер $\mu$ и $\nu$, если $m$ - мера, которая ялвяется Лебеговским продолжением $m_0$ с полукольца $\mathds{A} \times \mathds{B}$ на некоторую $\sigma$-алгебру $\mathds{A} \otimes \mathds{B}$.\\
	$m = \mu \times \nu$ - обозначение. \\
	$<X \times Y, \mathds{A} \otimes \mathds{B}, \mu \times \nu>$ - произведение пространств с мерой.

	\section{Сферические координаты в $ R^3 $ и в $ R^m $, их Якобианы}
	$x_1 = r \cdot \cos \phi_1$
    \hfill
	$1 \leq i \leq m-2: \phi_i \in [0,\pi]$

	$x_2 = r \cdot \sin \phi_1 \cdot \cos \phi_2$
    \hfill
	$i=m-1: \phi_i \in [0,2\pi]$

	$x_3 = r \cdot \sin \phi_1 \cdot \sin \phi_2 \cdot \cos \phi_3$
    \hfill
    $r \geq 0$

    $\vdots$\\
	$x_{m-2} = r \cdot \sin \phi_1 \cdot \sin \phi_2 \cdots \sin \phi_{m-3} \cdot \cos \phi_{m-2}$

	$x_{m-1} = r \cdot \sin \phi_1 \cdot \sin \phi_2 \cdots \sin \phi_{m-2} \cdot \cos \phi_{m-1}$

	$x_{m} = r \cdot \sin \phi_1 \cdot \sin \phi_2 \cdots \sin \phi_{m-2} \cdot \sin \phi_{m-1}$\\

	$\mathcal{J} = r^{m-1} \cdot (\sin \phi_1)^{m-2} \cdot (\sin \phi_2)^{m-3} \cdots (\sin \phi_{m-2})^{1} \cdot (\sin \phi_{m-1})^{0}$

	Что тут происходит идейно. Сначала мы проецируем наш $m$-мерный вектор на нормаль к $(m-1)$-мерной гиперплоскости. Потом рассматриваем проекцию на эту гиперплоскость и в ней рекурсивно повторяем процедуру, пока не дойдём до нашего любимого $\mathds{R}^2$. Уже в нём рассматривем обычные полярные координаты (отсюда и другие ограничения на размер угла).

	\section{Образ меры при отображении}
	$<X, \mathds{A}, \mu>$~--- пространство с мерой, $<Y, \mathds{B}, \_>$~--- пространство с $\sigma$-алгеброй.

	$\Phi: X \to Y$, $\Phi^{-1}(\mathds{B}) \subset \mathds{A}$ (прообраз любого множества из $\mathds{B}$ лежит в $\mathds{A}$).

	Пусть для $\forall E \in \mathds{B}$ $\nu(E) = \mu(\Phi^{-1}(E))$.

	$\nu$ является мерой на $Y$ и называется образом меры $\mu$ при отображении $\Phi$.

	\section{Взвешенный образ меры}
	$<X, \mathds{A}, \mu>$~--- пространство с мерой, $<Y, \mathds{B}, \_>$~--- пространство с $\sigma$-алгеброй.

	$\Phi: X \to Y$, $\Phi^{-1}(\mathds{B}) \subset \mathds{A}$ (прообраз любого множества из $\mathds{B}$ лежит в $\mathds{A}$).

	$\omega: X \to \overline{\mathds{R}}$, $\omega \geq 0$~--- измеримая.

	Пусть для $E \in \mathds{B}$ $\nu(E) = \int\limits_{\Phi^{-1}(E)} \omega~d\mu$.

	$\nu$ является мерой на $Y$ и называется взвешенным образом меры $\mu$.

	При $\omega \equiv 1$ взвешенный образ меры является обычным образом меры.

	\section{Плотность одной меры по отношению к другой}
	$<X, \mathds{A}, \mu>$~--- пространство с мерой.

	$\omega: X \to \overline{\mathds{R}}$, $\omega \geq 0$~--- измеримая.

	$\nu(E) = \int_E \omega(x)~d\mu$. $\nu$~--- мера на $X$.

	$\omega$ называется плотностью $\nu$ относительно $\mu$.

	\section{Заряд, множество положительности}
	\subsection{Заряд}
	$<X, \mathds{A}, \_>$~--- пространство с $\sigma$-алгеброй.

	$\phi: \mathds{A} \to \mathds{R}$ (конечная, не обязательно неотрицательная).

	$\phi$ счётно аддитивна.

	Тогда $\phi$~--- заряд.

	\subsection{Множество положительности}
	$A \subset X$~--- множество положительности, если
	$\forall B \subset A$, $B$ измеримо: $\phi(B) \geq 0$.

	\section{Интегральные неравества Гельдера и Минковского}
	$<X, \mathds{A}, \mu>$; $f, g : E \subset X \rightarrow \mathds{C}$ ($E$ - изм.)~--- заданы п.в, измеримы.\\
	\subsection{Нераветсво Гельдера}
	$p, q > 1 : \frac{1}{p} + \frac{1}{q} = 1$.
	\emph{Тогда:}
	${\displaystyle \int\limits_E |fg|d\mu \leq \left(\int\limits_E |f|^p d\mu\right)^\frac{1}{p} \cdot \left(\int\limits_E |g|^q d\mu)\right)^\frac{1}{q}}$
	\subsection{Нераверство Минковского}
	$1 \leq p < +\infty$.
	\emph{Тогда:}
	${\displaystyle \left(\int\limits_E |f + g|^p d\mu \right)^\frac{1}{p}
		\leq \left(\int\limits_E |f|^p d\mu\right)^\frac{1}{p}
		+ \left(\int\limits_E |g|^p d\mu\right)^\frac{1}{p}}$

	\section{Интеграл комплекснозначных функции}
	$(X, \mathds{A}, \mu)$ - пространство с мерой. $E \in \mathds{A}$
	
	$f: E \rightarrow \mathds{C}$
	
	$f$ измерима (суммируема), если $Im(f)$ и $Re(f)$ измеримы (суммируема)
	
	$\int_E f =\int_E Re(f) + i \cdot \int_E Im(f) $ 

	\section{Пространство $L_p(E,\mu),\ 1 \leq p < +\infty$}
	$<X, \mathds{A}, \mu>$, $E \in \mathds{A}$.\\
	$L_p'(E, \mu) = \{ f : \text{п.в.}\ E \rightarrow \mathbb{C},\ \text{изм.},\ \int\limits_E |f|^p d\mu < +\infty \}$\\
	Это линейное пространство (по нер-ву Минковского и линейности пространства измеримых функций).\\
	У этого пространства есть дефект~--- если определить норму как $||f|| = \left(\int\limits_E |f|^p\right)^\frac{1}{p}$, то будет сразу много нулей пространства (ненулевые функции, которые п.в. равны $0$, будут иметь норму $0$).
	Поэтому перейдем к фактор-множеству функций по отношению эквивалентности:\\
	$f \sim g$, если $f = g$ п.в.\\
	$ L_p(E, \mu) := L_p'(E, \mu) / \sim$ - лин. норм. пр-во с нормой $||f|| = \left(\int\limits_E |f|^p\right)^\frac{1}{p}$.\\

	\emph{NB1}: Его элементы --- классы эквивалентности обычных функций. Будем называть их тоже функциями. Они не умеют вычислять значение в точке (т.к. можно всегда подменить значение на любое другое и получить представителя все того же класса эквивалентности), но зато их можно интегрировать!\\

	\emph{NB2}: также иногда будем обозначать $||f||_p$ за норму $f$ в пространстве $L_p$.

	\section{Пространство $L_{\infty}(E,\mu)$}
	$L_\infty(E, \mu) =\{f : \text{п.в.}\ E \rightarrow \mathbb{C},\ \esssup\limits_E |f| < +\infty \}$\\
	\emph{NB1}: $||f||_\infty = \esssup\limits_E |f|$.\\

	\emph{NB2}: Новый вид нер-ва Гельдера : $||f \cdot g||_1 \leq ||f||_p \cdot ||g||_q$ (причем можно брать $p = +\infty, q = 1$ или наоборот).

	\section{Существенный супремум}
	$<X, \mathds{A}, \mu>$, $E \subset X$~--- изм., $f : \text{п.в.}\ E \rightarrow \overline{\mathbb{R}}$.\\

	\emph{Тогда}: $\esssup\limits_{x \in E} f(x) = \inf \{A \in R : f(x) \leq A\ \text{при п.в. $x$}\}$.

	В этом определении $A$ - существенная верхняя граница.

	\emph{Свойства}:
	\begin{enumerate}
		\item
		$\esssup\limits_E f \leq \sup\limits_E f$

		\item
		$f(x) \leq \esssup\limits_E f$ при п.в. $x \in E$.

		\item
		$\int\limits_E |fg|d\mu \leq \esssup\limits_E |g| \cdot \int\limits_E |f|d\mu$.
	\end{enumerate}

    \section{Условие $ L_{loc} $}

    $:($

    \section{Несобственный интеграл Лебега в $ R $}

    $:($

	\section{Фундаментальная последовательность, полное пространство}
	\subsection{Фундаментальная последовательность}
	$\{a_n\}$ - фунд. посл. в метрическом пр-ве $(X, \rho)$, если $\forall \epsilon > 0 \exists N: \forall n, k > N: \rho(a_n, a_k) < \epsilon$

	\subsection{Полное пространство}
	$X$ - полное пространство, если любая фундаментальная последовательность в нём сходится.

	\section{Плотное множество}
	Множество $A$ плотно во множестве $B$, если $\forall b \in B \ \forall \epsilon > 0$ верно, что $U_\epsilon(b) \cap A \neq \emptyset$.

\chapter{Теоремы}

\end{document}
