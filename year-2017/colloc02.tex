\documentclass[paper=a4, fontsize=14pt]{report}

\usepackage[russian]{babel}
\usepackage{scrextend}
\usepackage[utf8x]{inputenc}
\usepackage[T1,T2A]{fontenc}
\usepackage[left=1.5cm,right=1.5cm,top=1.5cm,bottom=1.5cm,bindingoffset=0cm]{geometry}
\usepackage[pdftex]{graphicx}
\usepackage{amsmath}
\usepackage{mathtools}
\usepackage{ulem}
\usepackage{mathrsfs}
\usepackage{amsfonts}
\usepackage{dsfont}
\usepackage{amssymb}
\usepackage{cmap}
\usepackage{hyperref}
\usepackage{tikz}

\DeclareMathOperator*{\esssup}{ess\, sup}
\DeclareMathAlphabet{\pazocal}{OMS}{zplm}{m}{n}

\parindent=0cm
\linespread{1.2}
\author{}

\title{Второй коллоквиум, семестр 4}

\begin{document}
\maketitle
\tableofcontents
\newpage

\chapter{Определения}
    \section{Равномерная сходимость несобственного интеграла}
    Ты проиграл

    \section{Нормальное топологическое пространство}
    Ты проиграл

    \section{Финитная функция}
    $\varphi : \mathbb{R}^m \rightarrow \mathbb{R}$. $\exists$ шар $B: \varphi \equiv 0 $ вне $B$. Тогда $\phi$~--- финитная.

    Множество непрерывных финитных функций обозначаем как $C_0(\mathbb{R}^m)$.

	\section{Гильбертово пространство}
	$\mathds{H}$~--- линейное пространство над $\mathds{R}$ или $\mathds{C}$, в котором задано скалярное произведение, и полное относительно соответствуйющей нормы, называется Гильбертовым.

	\section{Ортогональный ряд}
	$x_k \in \mathds{H}, \sum x_k$ называется ортогональным рядом, если $\forall k, l: k \neq l: x_k \bot x_l$.

	\section{Сходящийся ряд в Гильбертовом пространстве}
	$x_n \in \mathds{H}$.

	$\sum x_n$ сходится к $x$, если

	$S_n := \sum\limits_{k = 1}^n x_k$, $S_n \rightarrow x$ (то есть, $|S_n - x| \rightarrow 0$~--- сходимость по норме).

    \section{Ортогональная система (семейство) векторов}
	$\{e_k\} \in \mathds{H}$ - ортогональное семейство векторов, если $\forall k \neq l ~ e_k \bot e_l$, $\forall k\ e_k \neq 0$.

	\section{Ортонормированная система}
	$\{e_k\} \in \mathds{H}$ - ортонормированное семейство векторов, если ${e_k}$~--- ортогональное семейство векторов, и $\forall k ~ |e_k| = 1$.

	\section{Коффициенты Фурье}
	$\{e_k\}$ - ортогональное семейство векторов в $\mathds{H}, x \in \mathds{H}$.

	$c_k(x) = \dfrac{\langle x, e_k \rangle}{|e_k|^2}$ называются коэффициентами Фурье вектора $x$ по ортогональной системе $\{e_k\}$.

	\section{Ряд Фурье в Гильбертовом пространстве}

	$\sum c_k(x) \cdot e_k$ называется рядом Фурье вектора $x$ по ортогональной системе $\{e_k\}$.

	\section{Базис, полная, замкнутая ОС}

	$\{e_k\}$~--- ортогональная система в $\mathds{H}$.

	\begin{enumerate}

		\item $\{e_k\}$~--- \textbf{базис}, если $\forall x \in \mathds{H}\ \exists c_k$, что $x = \sum\limits_{k=1}^{+\infty} c_k \cdot e_k$

		\item $\{e_k\}$~--- \textbf{полная} О.С., если $(\forall k ~ z \bot e_k) \Rightarrow z = 0$.

		\item $\{e_k\}$~--- \textbf{замкнутая} О.С., если $\forall x \in \mathds{H} ~ \sum\limits_{k=1}^{+\infty} |c_k(x)|^2 \cdot ||e_k||^2 = ||x||^2$.

	\end{enumerate}

	\section{Измеримое множество на элементарной двумерной поверхности в $\mathds{R}^3$}
	
	$ M \subset R^3 $ -- простое 2-мерное многообразие, $ C^1 $ гладкости.

	$ \phi : \underset{\text{откр. обл.}}{O} \subset R^2 \rightarrow R^3$, $ \phi \in C^1 $ -- гомеофорфизм, $ \phi(O) = M $

	$ E \subset M $ -- изм. по Лебегу, если $ \phi^{-1}(E) $ -- изм. по Лебегу в $ R^2 $
	
	\section{Мера Лебега на простой двумерной поверхности в $\mathds{R}^3$}

	$ S(E) := \iint\limits_{\phi^{-1}(E)} | \phi_u' \times \phi_v'| dudv $ -- взвеш. образ меры Лебега отн. $ \phi $. Значит это мера на $ \mathbb{A}_{M} $

	\section{Поверхностный интеграл первого рода}
        
        $ M $ -- простое, гл, 2-мерное в $ R^3 $, $ \phi $ -- параметризация

	$ f $ -- изм. отн. $S$ (см. выше), $ f > 0 $ (или $ f $ -- суммируем. по $ S $)

        \emph{Тогда}: $ \int_M f dS$ -- называет инт. первого рода функ. $ f $ по поверхности $M$

	\section{Кусочно-гладкая поверхность в $\mathds{R}^3$}
	
        $M \subset \mathbb R^3$ называется кусочно-гладкой, если $M$ представляет собой объединение:

    \begin{itemize}
	    \item конечного числа простых гладких поверхностей
	    \item конечного числа простых гладких дуг
	    \item конечного числа точек
    \end{itemize}

	\section{Тригонометрический ряд}

	\begin{itemize}
		\item $$ \frac{a_0}{2}  + \sum_{k = 1}^{\infty} a_k\cos kx + b_k\sin kx $$
		(где $ a_i, b_i $ -- коэффициенты ряда).

		\item Другая форма:	$$ \sum_{k = \mathbb{Z}} c_k e^{ikx} $$

		Тогда $ S_n := \sum_{k = -N}^{N} c_k e^{ikx} $.
	\end{itemize}

	\section{Коэффициенты Фурье функции}

	\begin{itemize}
		\item $$ a_k(f) = \frac{1}{\pi} \int_{-\pi}^{\pi} f(x) \cos kx ~ dx $$

		\item $$ b_k(f) = \frac{1}{\pi} \int_{-\pi}^{\pi} f(x) \sin kx ~ dx $$

		\item $$ c_k(f) = \frac{1}{2\pi} \int_{-\pi}^{\pi} f(x) e^{-ikx} ~ dx $$

	\end{itemize}

    \section{Класс Липшица с константой M и показателем альфа}
    Ты проиграл

	\section{Сторона поверхности}

	Сторона (простой) гладкой двумерной поверхности {{---}} непрерывное поле единичных нормалей. Поверхность, для которой существует сторона, называется двусторонней. Если же стороны не существует, она называется односторонней.


	\section{Задание стороны поверхности с помощью касательных реперов}

	$F_1, F_2$ -- два касательных векторных поля к поверхности $M$.\\
	$\forall p \in M \quad F_1(p), F_2(p)$ -- Л.Н.З. касательные векторы.\\
	Тогда поле нормалей стороны определяется, как $n := F_1 \times F_2$\\

	Реп\'{е}р - пара векторов из $F_1 \times F_2$.

	\section{Интеграл II рода}

	$M$~--- простая гладкая двусторонняя двумерная поверхность в $\mathds{R}^3$.\\
	$n_0$~--- фиксированная сторона (одна из двух).\\
	$F : M \rightarrow \mathbb{R}^3$ -- векторное поле.\\

	\emph{Тогда} интегралом II рода назовем $\int\limits_{M} \langle F, n_0 \rangle ds$

	\emph{Замечания}
	\begin{enumerate}
		\item Смена стороны эквивалентна смене знака.
		\item Не зависит от параметризации.
		\item
		$F=(P, Q, R)$.

		Тогда интеграл имеет вид $\iint P dydz + Q dzdx + R dxdy$.

		\emph{NB:} $Q dxdz = -Q dzdx$.
	\end{enumerate}

	\section{Ориентация контура, согласованная со стороной поверхности}

	Ориентация контура согласована со стороной поверхности, если она задает эту сторону.\\

	\emph{Пояснение}:
	Рассмотрим некоторый контур (замкнутую петлю) и точку на нем. Построим два касательных вектора к контуру в этой точке: первый~--- снаружи от контура (задает направление <<движения>> по петле), второй~--- внутри контура. Тогда будем называть такую ориентацию согласованной со стороной, если направление векторного произведения первого и второго векторов в точке совпадает с направлением нормали к поверхности.

	\begin{center}
		\begin{tikzpicture}

			\draw (2,2) ellipse (3cm and 2cm);
			\draw[thick,->] (5,2) -- (5,3) node[anchor=south] {1};
			\draw[thick,->] (5,2) -- (2.4,2) node[anchor=south] {2};

		\end{tikzpicture}
	\end{center}

	\section{Ядро Дирихле, ядро Фейера}

	\subsection{Ядро Дирихле}

	$$ D_n(t) = \frac{1}{\pi}( \frac{1}{2} + \sum_{k = 1}^{n} \cos kt) $$

	\subsection{Ядро Фейера}

	$$ \Phi_n(t) = \frac{1}{n+1} \sum_{k = 0}^{n} D_k(t) $$

	\section{Свертка}

	$ f, K \in L_1[-\pi, \pi]$  -- пеорид.

	$$ (f \ast K)(x) = \int_{-\pi}^{\pi} f(x-t)K(t) dt$$

    \section{Аппроксимативная единица. (а. е.)}
        
        Пояснения: нужна 1-ца по свертке, но это не совсем функция, поэтому зададим её как предел последовательности.
        
        $ D \subset R, h_0 $ -- предельная точка $ D $ в $ \overline{R} $, 
        тогда $ \{K_h\}_{h \in D}$ -- а. е. если:
        \begin{itemize}
            \item[AE1: ] $ \forall h \in D ~~ K_h \in L_1[-\pi, \pi] ~ \int\limits_{-\pi}^{\pi} K_h = 1 $
            \item[AE2: ] $ \exists M ~ \forall h ~ \int\limits_{-\pi}^{\pi} |K_h| \leq M $
            \item[AE3: ] $ \forall \delta \in (0, \pi) ~  \int\limits_{E_\delta} | K_h | 
            \underset{h \rightarrow h_0}{\rightarrow} 0  $, где $E_{\delta} = [-\pi,\pi] \setminus [h_0 - \delta, h_0 + \delta]$
        \end{itemize}
        
    \section{Усиленная аппроксимативная единица.}
        
        Изменяем свойство AE3, на AE3': 
        
        $ \forall h ~ K_h \in L_\infty[-\pi, \pi]; ~ \forall \delta \in (0, \pi) ~ ~ \esssup\limits_{t \in E_\delta} |K_h(t)| \underset{h \rightarrow h_0}{\rightarrow} 0 $
        
    \section{Метод суммирования средними арифметическими}
        $\sum a_n = \lim\limits_{n \to \infty} \dfrac{1}{n+1} \cdot \sum\limits_{k=0}^n S_k$
    \section{Суммы Фейера.}
        $ \sigma_n = \frac{1}{n + 1} \sum\limits_{k=0}^{n} S_k(f(x)) = \int\limits_{-\pi}^{\pi} f(x - t) \Phi_n(t) dt$

	\section{Ротор, дивергенция векторного поля}
    $F = (P, Q, R)$~--- векторное поле в $\mathds{R}^3$.

	$ rot\ F = (R'_y - Q'_z, P'_z - R'_x, Q'_x - P'_y) $ -- ротор, вихрь

	$div\ F = P'_x + Q'_y + R'_z$. Многомерный случай определяется аналогично.

	\section{Соленоидальное векторное поле}
	Векторное поле $A$~--- соленоидальное, если $\exists$ векторное поле $B: rot\ B = A$. Тогда $B$ называется векторным потенциалом поля $A$.

	\section{Бескоординатное определение ротора и дивергенции}
	$rot\ F$~--- это такое векторное поле, что $\forall a \ \forall n_0 (rot F(a))_{n_0} = \lim\limits_{r\to 0} \frac{1}{\pi r^2} \int\limits_{\partial B_r} F_ldl$
		
	где $ B_r $ -- круговой контур, $ n_0 $ -- нормаль контура, $ F_l $ -- проекция на касательное направление контура. 

	Пояснение: $ \frac{1}{\pi r^2} \int\limits_{\partial B_r} F_ldl =  \frac{1}{\pi r^2} \iint\limits_{B_r} \langle rot\ F, n_0 \rangle dS \underset{r \approx 0}{\approx} rot F(a)$
	


	$div F(a) = \lim\limits_{r\to 0} \frac{1}{\lambda_3(B(a,r))} \iiint\limits_{B(a,r)} div F \,dx\,dy\,dz = \lim\limits_{r\to 0} \frac{1}{\lambda_3(B(a,r))} \iint\limits_{\partial B(a,r)} \langle F, n_0 \rangle dS$

\chapter{Теоремы}
    \section{Перестановка двух предельных переходов. Предельный переход в несобственном интеграле.}
    Ты проиграл

    \section{Вычисление интеграла Дирихле}
    Ты проиграл

    \section{Теорема об интегрировании несобственного интеграла по параметру}
    Ты проиграл

    \section{Правило Лейбница для несобственных интегралов}
    Ты проиграл

    \section{Теорема о вычислении интеграла по мере Бореля---Стилтьеса (с леммой)}
    Ты проиграл

    \section{Плотность в $L^p$ множества ступенчатых функций}
    Ты проиграл

    \section{Лемма Урысона}
    $X$~--- нормальное топологическое пространство, то есть:
    \begin{enumerate}
        \item Все одноточечные множества замкнуты.
        \item Любые два непересекающихся замкнутых множества отделимы окрестностями:\\
        $A,B$~--- замкнуты, $A \cap B = \emptyset$ $\Rightarrow$ $\exists A_1,B_1$~--- открыты, $A_1 \cap B_1 = \emptyset$,
        $A \subset A_1$, $B \subset B_1$.
    \end{enumerate}

    $F_0$, $F_1$~--- замкнуты, $F_0 \cap F_1 = \emptyset$.

    \emph{Тогда:} $\exists f: X \rightarrow [0,1]$, непрерывная (в смысле топологического определения непрерывности), равная $0$ на $F_0$ и равная $1$ на $F_1$.

    \section{Плотность в $L^p$ множества финитных непрерывных функций}
    $(\mathbb{R}^m, \mathbb{A}, \lambda_m)$\\

    $E \subset \mathbb{R}^m -$ изм. Тогда множество финитных непрерывных функций плотно в $L_p(E, \lambda_m), p \in [1; +\infty]$\\

    \section{Теорема о непрерывности сдвига}
    \emph{Обозначения}:

    $f_h := f(x+h)$ \\
    $[0, T] \subset \mathbb{R}$. Будем считать, что $L_p[0, T]$ состоит из $T$-периодических функций $\mathbb{R} \rightarrow \overline {\mathbb{R}}$. Отсюда $\int_{0}^{T} f = \int_{a}^{a+T} f.$ \\
    $\widetilde{C}[0, T] = {f \in C[0, T]: f(0) = f(T) }. ||f|| = max_{x\in[0,T]}|f(x)|$ \\
    NB: $f \in \widetilde{C}[0, T] \Rightarrow f$ равномерно непрерывна (по т. Кантора).

    \emph{Формулировка}:
    \begin{enumerate}
        \item $f - $ рвнм. непр. на $\mathbb{R}^m$. Тогда $||f-f_h||_\infty \rightarrow 0$ при $h\rightarrow 0$.
        \item $1 \leq p < + \infty \ f \in L_p(\mathbb{R}^m, \lambda_m)$. Тогда $||f-f_h||_p \rightarrow 0$.
        \item $f \in \widetilde{C}[0, T]$. Тогда $||f-f_h||_\infty \rightarrow 0$.
        \item $1 \leq p < + \infty \  f \in L_p[0; T]$. Тогда $||f-f_h||_p \rightarrow 0$.
    \end{enumerate}

    \section{Теорема о свойствах сходимости в гильбертовом пространстве}
    \begin{enumerate}
        \item $x_n \rightarrow x, y_n \rightarrow y \Rightarrow \langle x_n, y_n \rangle \rightarrow \langle x, y \rangle$

        \item $\sum x_k$ сходится, тогда $\forall y: \sum \langle x_k, y \rangle = \langle \sum x_k, y \rangle$

        \item $\sum x_k$ - ортогональный ряд, тогда $\sum x_k$ - сх $\Leftrightarrow \sum |x_k|^2$ сходится, при этом $|\sum x_k|^2 = \sum |x_k|^2$

    \end{enumerate}

    \section{Теорема о коэффициентах разложения по ортогональной системе}

    $\{e_k\}$ {{---}} ортогональная система в $\mathds{H},\ x \in \mathds{H}, x = \sum\limits_{k=1}^{+\infty} c_k \cdot e_k$

    \emph{Тогда:}
    \begin{enumerate}

        \item $\{e_k\}$ {{---}} Л.Н.З.

        \item $c_k = \dfrac{\langle x, e_k \rangle}{||e_k||^2}$

        \item $c_k \cdot e_k$ {{---}} проекция $x$ на прямую $\{te_k \mid t \in \mathbb{R}\ (\text{или}\ \mathbb{C})\}$\\
        Иными словами, $x = c_k \cdot e_k + z$, где $z \bot e_k$

    \end{enumerate}

    \section{Теорема о свойствах частичных сумм ряда Фурье. Неравенство Бесселя}

    $\{e_k\}$ {{---}} ортогональная система в $\mathds{H},\ x \in \mathds{H}, n \in \mathbb{N}$

    $S_n = \sum\limits_{k=1}^{n} c_k(x)e_k,\ \pazocal{L} = Lin(e_1, e_2, \ldots e_n) \subset \mathds{H}$

    \emph{Тогда:}

    \begin{enumerate}

        \item $S_n$ {{---}} орт. проекция $x$ на пр-во $\pazocal{L}$. Иными словами $x = S_n + z,\ z \bot \pazocal{L}$

        \item $S_n$ {{---}} наилучшее приближение $x$ в $\pazocal{L}\ (||x - S_n|| = \min\limits_{y \in \pazocal{L}} ||x - y||)$

        \item $||S_n|| \leqslant ||x||$

    \end{enumerate}

    \section{Теорема Рисса -- Фишера о сумме ряда Фурье. Равенство Парсеваля}
    $\{e_k\}$ -- орт. сист. в $\mathds{H}$, $x \in \mathds{H}$\\

    \emph{Тогда}:
    \begin{enumerate}
        \item Ряд Фурье $\sum\limits_{k=1}^{+\infty} c_k(x) e_k$ сходится в $\mathds{H}$
        \item $x =\sum\limits_{k=1}^{+\infty} c_k e_k + z \Rightarrow \forall k \ z \bot e_k$
        \item $x =\sum\limits_{k=1}^{+\infty} c_k e_k \Leftrightarrow \sum\limits_{k=1}^{+\infty} \vert c_k \vert^2 \|e_k\|^2=\|x\|^2$
    \end{enumerate}

    \section{Теорема о характеристике базиса}

    $\{e_k\}$~--- ортогональная система в $\mathds{H}$\\

    \emph{Тогда} эквивалентны следующие утверждения:
    \begin{enumerate}
        \item $\{e_1\}$~--- базис.
        \item $\forall x, y \in \mathds{H} \quad \langle x, y \rangle = \sum c_k(x)\overline{c_k(y)}\|e_k\|^2$ (обобщенное уравнение замкнутости)
        \item $\{e_k\}$~--- замкнутая система.
        \item $\{e_k\}$~--- полная система.
        \item $Lin(e_1, e_2, \mathellipsis)$~--- плотна в $\mathds{H}$
    \end{enumerate}

    \section{Лемма о вычислении коэффициентов тригонометрического ряда}

    Пусть $S_n \rightarrow f$ в $L_1(-\pi, \pi]$\\

    \emph{Тогда}:

    $a_k = \frac{1}{\pi}\int\limits_{-\pi}^{\pi}f(x)\cos kx\ dx \quad k = 0, 1, 2, \mathellipsis$

    $b_k = \frac{1}{\pi}\int\limits_{-\pi}^{\pi}f(x)\sin kx\ dx \quad k = 0, 1, 2, \mathellipsis$

    $c_k = \frac{1}{2\pi}\int\limits_{-\pi}^{\pi}f(x)e^{-ikx}\ dx \quad k = 0, 1, 2, \mathellipsis$

    \section{Теорема Римана-Лебега}
    $E \subset \mathds{R}^1$ -- измеримо\\ $f \in L_1(E, \lambda), ~ \lambda \text{- мера Лебега}$ \\
    \emph{Тогда:}
    $$\int\limits_{E}f(x)e^{ikx}dx \xrightarrow[k \to +\infty]{} 0$$

    $$\int\limits_{E}f(x)cos(kx)dx \xrightarrow[k \to +\infty]{} 0$$

    $$\int\limits_{E}f(x)sin(kx)dx \xrightarrow[k \to +\infty]{} 0$$
    \section{Три следствия об оценке коэффициентов Фурье}
    Ты проиграл

    \section{Принцип локализации Римана}

    $ f, G \in L_1[-\pi, \pi] $\\
    $ x_0 \in R, \delta > 0 $\\
    $ f \equiv g $ на $ (x_0 - \delta, x_0 + \delta) $

    \emph{Тогда:}

    $ S_n(f, x_0) - S_n(g, x_0) \to 0 $

    \section{Признак Дини. Следствия}

    $ f \in L_1[-\pi, \pi] $\\
    $ x_0 \in R $\\
    $ S \in R $

    (*) $ \int\limits_{0}^{\pi} \frac{|f(x_0 + t) - 2S + f(x_0 - t)|}{t} dt $ сходится

    \emph{Тогда:}

    $ S_n(f, x_0) \to S $\\

    \emph{Следствие1:}

    $ \exists $ 4 предела\\
    $ \lim_{t \to \pm 0} \frac{f(x_0 + t) - f(x_0 \pm 0)}{t} $

    \emph{Тогда:}

    Ряд фурье сходится в $ x_0 $ как $ \frac{f(x_0 + 0) + f(x_0 - 0)}{2} $\\

    \emph{Следствие2:}

    $ f \in L_1[-\pi, \pi] $\\
    $ f $ непрерывна в  $ x_0 $\\
    $ \exists $ конечные $ f_{+}'(x_0), f_{-}'(x_0) $

    \emph{Тогда:}

    $ S_n(f, x_0) \to f(x_0) $

    \section{Корректность определения свертки}

    $ f, K \in L_1[-\pi, \pi] $

    \emph{Тогда:} $ (f * K)$ -- корректно заданная фукнция из $ L_1[-\pi, \pi] $

    \section{Свойства свертки функции из $ L^p $ с фукнцией из $ L^q $}

    $ f \in L^p $; $ K \in L^q $

    $ 1 \leqslant p \leqslant +\infty $; $ \frac{1}{p} + \frac{1}{q} = 1 $

    \emph{Тогда:}
    \begin{itemize}
        \item $ f \ast K $ -- непр. на $ [-\pi, \pi] $
        \item $ ||f \ast K||_{\infty} \leqslant ||K||_q ||f||_p $
    \end{itemize}

    \section{Теорема о свойствах аппроксимативной единицы}
    \begin{enumerate}
        \item $f \in \widetilde{C}[-\pi, \pi] \Rightarrow (f * K_h) \underset{h \rightarrow h_0}{\rightrightarrows} f $, где
        свертка $(f* K)(x) = \int_{-\pi}^{\pi} f(x-t)K(t)dt$
        
        \item $f \in L^1 [-\pi, \pi] \Rightarrow ||(f * K_h) - f||_1 \underset{h \rightarrow h_0}{\rightarrow} 0 $
        \item $K_h$ - усил. апрокс ед. $f$ - непр. в точке $x$. Тогда $(f * K_h)(x) \rightarrow f(x)$

        \emph{Замеч.}) пункт 2 верен для $ L_p $
    \end{enumerate}

    \section{Теорема Фейера}
        \begin{enumerate}
            \item $ f \in \widetilde{C}[\pi, -\pi] $, тогда $ \sigma_n \underset{n \rightarrow +\infty}{\rightrightarrows} f$

            \item $ f \in L_p[\pi, -\pi], (1 \leq p < +\infty)$, тогда $ ||\sigma_n(f) - f||_p \underset{n \rightarrow +\infty}{\rightarrow} 0 $

            \item $ f \in L_1[\pi, -\pi], f $- непр. в т. $x$, тогда $ \sigma_n(f, x) \underset{n \rightarrow +\infty}{\rightarrow} f(x) $
        \end{enumerate}

    \section{Полнота тригонометрической системы и другие следствия теоремы Фейера}
    Ты проиграл

    \section{Лемма об оценке интеграла ядра Дирихле}
    \begin{enumerate}
        \item $D_n(t) = \frac{\sin nt}{\pi t} + \frac{1}{2\pi}(\cos nt + h(t)\sin nt)$, где $h(t)$ не зависит от n и $|h(t)| \leq 1$
            на $[-\pi;\pi]$.
        \item $\forall x, |x| < 2\pi\ |\int_0^x D_n(t) dt| < 2$
    \end{enumerate}
        \section{Формула Грина}
        $D \subset \mathds{R}^2$ -- компакт, связное, одновясвязное, ориентировано\\$\delta D-C^2\text{-гладкая кривая, тоже ориентировано}$\\
        $D$ и $\delta D$ ориентированы согласовано \\
        $P, Q$ -- функции, гладкие в открытой области $O \supset D$ \\
        \emph{Тогда:} $$\iint\limits_{D}(\frac{\delta Q}{\delta x} - \frac{\delta P}{\delta y})dxdy = \int\limits_{\delta D}(P(x,y)dx + Q(x,y))dy$$

    \section{Теорема об интегрировании ряда Фурье}
    $f \in L_1[-\pi;\pi]$.

    Тогда $\forall a, b \in \mathbb{R}$:
    $$\int_a^b f(x)dx = \sum_{k\in\mathbb{Z}} c_k(f) \int_a^b e^{ikx} dx$$
    Сумма по $k \in \mathbb{Z}$ понимается в смысле главного значения ($\lim_{n \to \infty} \sum_{k=-n}^n$).

    \emph{Замечание:} Ряд Фурье $f$ может всюду расходиться, но ряд интеграла всегда сходится.
    \section{Следствие о синус-коэффициентах интегрируемой функции}
    Ты проиграл
    
    \section{Лемма о слабой сходимости сумм Фурье}
    Ты проиграл
    
    \section{Леммы о равномерной ограниченности сумм Фурье и об обобщенном равенстве Парсеваля}
    Ты проиграл
    
    \section{Формула Стокса}
    $\Omega$ -- эллиптическая, гладкая, двусторонняя поверхность, $C^2-$гладкое; $n_0$ -- сторона\\
    $\delta \Omega$ - ориентирована согласовано с $n_0$\\
    $(P,Q,R)$ -- векторное поле на $\Omega$, заданное в $O$ - откр. $: \Omega \subset O\subset \mathds{R}^3$ \\
    \emph{Тогда:} $$\int\limits_{\delta \Omega}(Pdx + Qdy+Rdz) =
    \iint\limits_{\Omega}((R^{'}_y- Q{'}_z)dydz
    +(P^{'}_z-R^{'}_x)dzdx + (Q^{'}_x-P^{'}_y)dxdy)$$

    \section{Формула Гаусса--Остроградского}
    $V = \{(x,y,z)\in \mathbb{R}^3: (x,y) \in G, f(x, y) \leq z \leq F(x,y)\}, G \subset \mathbb{R}^2, \partial G -$ гладкая кривая в $\mathbb{R}^2, F \in "C'(G)"$ (кавычки означают "включая границу, то есть с более широкой гладкой областью"), $\partial V - $ внешняя сторона, $R: O(V) \rightarrow \mathbb{R}$. Тогда
    $$\iiint\limits_{V}{\frac {\partial R}{\partial z} \,dx\,dy\,dz = \iint\limits_{\partial V}{R\,dx\,dy}}$$

    \section{Соленоидальность бездивергентного векторного поля}

    $A$ - соленоидально $ \Leftrightarrow div(A) = 0 $

    $A \in C^1, O $ --- хорошая область.
\end{document}
